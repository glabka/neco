\documentclass[11pt,a4paper]{article}
\usepackage[czech]{babel}
\usepackage[utf8]{inputenc}
%\usepackage{siunitx}
\usepackage{amsmath}
\usepackage{float} %kvůli umístění tabulek a podobně.
\usepackage{graphicx}
%\usepackage[]{mcode} %pro matlabovský kód

\title{%
  Laboratorní úloha \\
  \large Rotační kyvadlo}
\author{Tomáš Glabazňa, Matouš Vrba}
\date{\today}

\begin{document}
\maketitle

%\newcommand{\xjd}{\dot{x}_1} %o == one == 1, číslice v příkazu totiž být nemůžou
\newcommand{\xd}[1]{\dot{x}_{#1}} %př. \xd{1} == \xjd
%\newcommand{\phimd}{\dot{\varphi}_m}
\newcommand{\phid}[1]{\dot{\varphi}_{#1}}
\newcommand{\phin}[1]{\varphi_{#1}} %phi normální (nederivované)

\section*{Rovnice}
$$
\xd{1} = \phid{m}
$$
$$
\xd{2} = \phid{p}
$$

%%%%%%%%%%%%%%%%Rovnice ze zadání (přepsané pro naše stavy)%%%%%%%%%%%
%Zabaleni sin a cos bych se mohl rozhodnout, zda bude argument v závorce, či ne.
\newcommand{\coss}[1]{\cos{#1}}
\newcommand{\sinn}[1]{\sin{#1}}
\newcommand{\sinnN}[1]{\sin^2{#1}} %N = nadruhou
$$
k_1 = J_m + m r^2
$$
$$
k_2 = m l r
$$
$$
k_3 = m g l
$$
$$
\xd{3} =
\frac{
	-J_p k_2 \coss{x_2} x_4^2 + 2 \delta k_2 \sinn{x_2} x_4 - J_p M + J_p b x_3 	+ k_2 k_3 \coss{x_2} \sinn{x_2}
}{
 	-J_p k_1 + k_2^2 \sinnN{x_2}
}
$$
$$
\xd{4} =
\frac{
	2 \delta x_4 + k_3 \coss{x_2} - \frac{k_2^2}{k_1} \sinn{x_2} \coss{x_2} 			 	x_4^2 - \frac{k_2}{k_1} \sinn{x_2} M + \frac{k_2 b}{k_1} \sinn{x_2} x_3
}{
	-J_p + \frac{k_2^2}{k_1} \sinnN{x_2}
}
$$
%%%%%%%%%%%%%%%%%%%%%%%%%%%%%%%%%%%

$$
x_1 = \phin{m}
$$
$$
x_2 = \phin{p}
$$
$$
x_3 = \phid{m}
$$
$$
x_4 = \phid{p}
$$
$$
y_1 = x_1
$$
$$
y_2 = x_2
$$


\end{document}