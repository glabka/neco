\documentclass[11pt,a4paper]{article}
\usepackage[czech]{babel}
\usepackage[utf8]{inputenc}
%\usepackage{siunitx}
\usepackage{amsmath}
\usepackage{float} %kvůli umístění tabulek a podobně.
\usepackage{graphicx}
%\usepackage[]{mcode} %pro matlabovský kód
\usepackage{etoolbox} %kvůli strequal v newcommand


\title{%
  Laboratorní úloha - modelovací část\\
  \large Rotační kyvadlo}
\author{Tomáš Glabazňa, Matouš Vrba}
\date{\today}

\begin{document}
\maketitle

\clearpage

\section{Rovnice}
%\newcommand{\u}{\u} %vstup M jsme si vybrali jako vstup, tedy M = u = k_u * u_norm, kde k_u je konstanta, u_norm to, co v matlabu nastavujeme. 

\newcommand{\x}[2][]{ %první argument je nepovinný, tj. v těchto závorkách []. Pokud to je [t], tak se ta proměnná zobrazí jako funkce času
  \ifstrequal{#1}{t} %podmínka - pokud je první volitelná argument t
   	{x_{#2}(t)} % pak udělej toto
    {x_{#2}}    % jinak udělej toto
}
\newcommand{\xd}[2][]{
  \ifstrequal{#1}{t} 
   	{\dot{x}_{#2}(t)}
    {\dot{x}_{#2}}
}
\newcommand{\y}[1]{y_{#1}}
\newcommand{\M}{M}
\newcommand{\phid}[2][]{
	\ifstrequal{#1}{t}
	{\dot{\varphi}_{#2}(t)}
	{\dot{\varphi}_{#2}}
	}
\newcommand{\phin}[2][]{ %phi normální (nederivované)
	\ifstrequal{#1}{t}	
	{\varphi_{#2}(t)}
	{\varphi_{#2}}
	}
\newcommand{\coss}[1]{\cos{(#1)}}
\newcommand{\sinn}[1]{\sin{(#1)}}
\newcommand{\sinnN}[1]{\sin^2{(#1)}} %N = nadruhou

Stavy systému jsme si definovali jako:
\begin{align*}
& \x{1} = \phin[t]{m}		 && 		\x{2} = \phin[t]{p} \\
& \x{3} = \phid[t]{m}		 &&		\x{4} = \phid[t]{p} \\
\end{align*}

Výstupy jsme definovali jako:
\begin{align*}
& \y{1} = \phin[t]{m}		 &&		\y{2} = \phin[t]{p} \\
\end{align*}

Stavové rovnice jsou tedy:

\begin{align*}
& \xd{1} = \phid{m}    &&	\xd{2} = \phid{p} \\
\end{align*}
\begin{figure}[H]
\vspace*{-1.3 cm}
$$
\xd{3} =
\frac{
	-J_p k_2 \coss{\x{2}} \x{4}^2 + 2 \delta k_2 \sinn{\x{2}} \x{4} - J_p \M + J_p b \x{3} 	+ k_2 k_3 \coss{\x{2}} \sinn{\x{2}}
}{
 	-J_p k_1 + k_2^2 \sinnN{\x{2}}
}
$$
$$
\xd{4} =
\frac{
	2 \delta \x{4} + k_3 \coss{\x{2}} - \frac{k_2^2}{k_1} \sinn{\x{2}} \coss{\x{2}} 			 	\x{4}^2 - \frac{k_2}{k_1} \sinn{\x{2}} M + \frac{k_2 b}{k_1} \sinn{\x{2}} \x{3}
}{
	-J_p + \frac{k_2^2}{k_1} \sinnN{\x{2}}
}
$$
\end{figure}
\begin{figure}[H]
\vspace*{-1.3 cm}
\begin{align*}
& \y{1} = \x{1}			 &&		\y{2} = \x{2} \\
\end{align*}
\end{figure}

, kde pro konstanty $k_1$, $k_2$ a $k_3$ platí:
$$
k_1 = J_m + m r^2
$$
$$
k_2 = m l r
$$
$$
k_3 = m g l
$$


\end{document}