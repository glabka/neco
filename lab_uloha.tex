\documentclass[11pt,a4paper]{article}
\usepackage[czech]{babel}
\usepackage[utf8]{inputenc}
%\usepackage{siunitx}
\usepackage{amsmath}
\usepackage{float} %kvůli umístění tabulek a podobně.
\usepackage{graphicx}
%\usepackage[]{mcode} %pro matlabovský kód

\title{%
  Laboratorní úloha \\
  \large Rotační kyvadlo}
\author{Tomáš Glabazňa, Matouš Vrba}
\date{\today}

\begin{document}
\maketitle


\section*{Rovnice}
\newcommand{\x}[1]{x_{#1}}
\newcommand{\xd}[1]{\dot{x}_{#1}}
\newcommand{\M}{M}
\newcommand{\phid}[1]{\dot{\varphi}_{#1}}
\newcommand{\phin}[1]{\varphi_{#1}} %phi normální (nederivované)
\newcommand{\coss}[1]{\cos{(#1)}}
\newcommand{\sinn}[1]{\sin{(#1)}}
\newcommand{\sinnN}[1]{\sin^2{(#1)}} %N = nadruhou

$$
\xd{1} = \phid{m}
$$
$$
\xd{2} = \phid{p}
$$

%%%%%%%%%%%%%%%%Rovnice ze zadání (přepsané pro naše stavy)%%%%%%%%%%%
%Zabaleni sin a cos bych se mohl rozhodnout, zda bude argument v závorce, či ne.
$$
k_1 = J_m + m r^2
$$
$$
k_2 = m l r
$$
$$
k_3 = m g l
$$
$$
\xd{3} =
\frac{
	-J_p k_2 \coss{\x{2}} \x{4}^2 + 2 \delta k_2 \sinn{\x{2}} \x{4} - J_p \M + J_p b \x{3} 	+ k_2 k_3 \coss{\x{2}} \sinn{\x{2}}
}{
 	-J_p k_1 + k_2^2 \sinnN{\x{2}}
}
$$
$$
\xd{4} =
\frac{
	2 \delta \x{4} + k_3 \coss{\x{2}} - \frac{k_2^2}{k_1} \sinn{\x{2}} \coss{\x{2}} 			 	\x{4}^2 - \frac{k_2}{k_1} \sinn{\x{2}} M + \frac{k_2 b}{k_1} \sinn{\x{2}} \x{3}
}{
	-J_p + \frac{k_2^2}{k_1} \sinnN{\x{2}}
}
$$
%%%%%%%%%%%%%%%%%%%%%%%%%%%%%%%%%%%

$$
x_1 = \phin{m}
$$
$$
x_2 = \phin{p}
$$
$$
x_3 = \phid{m}
$$
$$
x_4 = \phid{p}
$$
$$
y_1 = x_1
$$
$$
y_2 = x_2
$$


\end{document}