\documentclass[a4paper,12pt]{article}

\usepackage[czech]{babel}
\usepackage[utf8]{inputenc}

\usepackage{graphicx}	%kvůli pdf souboru z MATLABu
\usepackage{amsmath}	%matice
\usepackage{setspace}	%rozteče v maticích
\usepackage{float}		%obrázky
\usepackage{verbatim}   %víceřádkové komentáře
\usepackage{siunitx}	%hezký jednotky

\sisetup{locale = DE}

\title{Simulační úloha - Servo Amira}
\author{Matouš Vrba}
\date{\today}

% === Formát stránky ===
\usepackage[a4paper]{geometry}
\geometry{
	verbose,
	tmargin=2.2cm,
	bmargin=1.5cm,
	lmargin=1.5cm,
	rmargin=1.5cm}

\begin{document}

\maketitle
\pagebreak
\section{Linearizace modelu}
Stabilní rovnovážná poloha kyvadla je v pracovním bodě $\mathbf{x_0} = [x_{1p}, x_{2p}, x_{3p}, x_{4p}] = [0, -\frac{\pi}{2}, 0, 0]$, ${u_0} = 0$.
\newline
Matice linearizovaného systému v pracovním bodě $\mathbf{x_0}, u_0$:

\renewcommand{\arraystretch}{1.3}
{\Large
\begin{align*}
&A = 
\begin{pmatrix}
\frac{\partial x_1}{\partial x_1} & \frac{\partial x_1}{\partial x_2} & \frac{\partial x_1}{\partial x_3} & \frac{\partial x_1}{\partial x_4}	\\
\frac{\partial x_2}{\partial x_1} & \frac{\partial x_2}{\partial x_2} & \frac{\partial x_2}{\partial x_3} & \frac{\partial x_2}{\partial x_4}	\\
\frac{\partial x_3}{\partial x_1} & \frac{\partial x_3}{\partial x_2} & \frac{\partial x_3}{\partial x_3} & \frac{\partial x_3}{\partial x_4}	\\
\frac{\partial x_4}{\partial x_1} & \frac{\partial x_4}{\partial x_2} & \frac{\partial x_4}{\partial x_3} & \frac{\partial x_4}{\partial x_4}
\end{pmatrix}_{\biggr\rvert_\mathbf{x_0}} =
\left(\begin{array}{cccc} 0 & 0 & 1 & 0\\ 0 & 0 & 0 & 1\\ 0 & \frac{\mathrm{k_2}\, \mathrm{k_3}}{\mathrm{J_p}\, \mathrm{k_1} - {\mathrm{k_2}}^2} & -\frac{\mathrm{J_p}\, b}{\mathrm{J_p}\, \mathrm{k_1} - {\mathrm{k_2}}^2} & \frac{2\, \mathrm{\delta}\, \mathrm{k_2}}{\mathrm{J_p}\, \mathrm{k_1} - {\mathrm{k_2}}^2}\\ 0 & -\frac{\mathrm{k_3}}{\mathrm{J_p} - \frac{{\mathrm{k_2}}^2}{\mathrm{k_1}}} & \frac{b\, \mathrm{k_2}}{\mathrm{k_1}\, \left(\mathrm{J_p} - \frac{{\mathrm{k_2}}^2}{\mathrm{k_1}}\right)} & -\frac{2\, \mathrm{\delta}}{\mathrm{J_p} - \frac{{\mathrm{k_2}}^2}{\mathrm{k_1}}} \end{array}\right)	\\ \\
&B =
\begin{pmatrix}
\frac{\partial x_1}{\partial u}	\\
\frac{\partial x_2}{\partial u}	\\
\frac{\partial x_3}{\partial u}	\\
\frac{\partial x_4}{\partial u}
\end{pmatrix}_{\biggr\rvert_{u_0}} =
\left(\begin{array}{c} 0\\ 0\\ \frac{\mathrm{J_p}}{\mathrm{J_p}\, \mathrm{k_1} - {\mathrm{k_2}}^2}\\ -\frac{\mathrm{k_2}}{\mathrm{k_1}\, \left(\mathrm{J_p} - \frac{{\mathrm{k_2}}^2}{\mathrm{k_1}}\right)} \end{array}\right)	\\ \\
&C =
\begin{pmatrix}
\frac{\partial y_1}{\partial x_1} & \frac{\partial y_1}{\partial x_2} & \frac{\partial y_1}{\partial x_3} & \frac{\partial y_1}{\partial x_4}	\\
\frac{\partial y_2}{\partial x_1} & \frac{\partial y_2}{\partial x_2} & \frac{\partial y_2}{\partial x_3} & \frac{\partial y_2}{\partial x_4}
\end{pmatrix}_{\biggr\rvert_\mathbf{x_0}} =
\left(\begin{array}{cccc} 1 & 0 & 0 & 0\\ 0 & 1 & 0 & 0 \end{array}\right)	\\ \\
&D =
\begin{pmatrix}
\frac{\partial y_1}{\partial u}	\\
\frac{\partial y_2}{\partial u}
\end{pmatrix}_{\biggr\rvert_{u_0}} =
\left(\begin{array}{c} 0\\ 0 \end{array}\right)
\end{align*}
}

\newpage
\section{Saturace, pásma necitlivosti, apod.}
Vstup systému v MATLABu je normalizován na interval $u_{norm} \in \left<-1; 1\right>$, což vymezuje saturaci vstupu. Pásmo necitlivosti jsme identifikovali v intervalu $\left<-0.007; 0.007\right>$.
\newline
\newline
Při výchylce kyvadla v rozmezí $\pm \SI{12}{\degree}$ je vliv kyvadla na polohu kyvadla zanedbatelný a dá se považovat za pásmo necitlivosti. Toho jsme využili také při identifikaci dynamiky kyvadla z počátečních podmínek.
\newline
\newline
Další nelinearitu jsme identifikovali u ramene, které se na jednu stranu posouvalo snáze, než na druhou.

\section{Identifikace dynamiky motoru (ramene)}



























\end{document}