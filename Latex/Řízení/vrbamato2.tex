\documentclass[11pt,a4paper]{article}
\usepackage[czech]{babel}
\usepackage[utf8]{inputenc}
\usepackage{amsmath}
\usepackage{float} %kvůli umístění tabulek a podobně.
%\usepackage[]{mcode} %pro matlabovský kód
\usepackage{etoolbox} %kvůli strequal v newcommand
\usepackage[czech]{babel}
\usepackage[utf8]{inputenc}

\usepackage{graphicx}	%kvůli pdf souboru z MATLABu
\usepackage{caption}		%kvůli umístění několika obrázku do jednoho figuru
\usepackage{subcaption} %kvůli umístění několika obrázku do jednoho figuru
\usepackage{setspace}	%rozteče v maticích
\usepackage{verbatim}   %víceřádkové komentáře
\usepackage{siunitx}	%hezký jednotky
\usepackage{epstopdf}

\sisetup{locale = DE} % kvůli desetinné čárce (př. v siunits.)


% === Formát stránky ===
\usepackage[a4paper]{geometry}
\geometry{
	verbose,
	tmargin=2.2cm,
	bmargin=1.5cm,
	lmargin=1.5cm,
	rmargin=1.5cm}

\title{%
  Laboratorní úloha - řídící část\\
  \large Rotační kyvadlo}
\author{Tomáš Glabazňa, Matouš Vrba}
\date{\today}

\begin{document}
\maketitle

\clearpage

\section{Regulátory polohy ramene}
\begin{figure}[H]
\centering
\includegraphics[width=0.7\textwidth]{schema_rizeni_nanecisto.jpg}
\caption{Zapojení pro testování regulátorů}
\label{ram_p_1}
\end{figure}

\subsection{Proporční regulátor}

Proporční zpětnovazební regulátor jsme navrhovali přímo na modelu a jeho vhodnou hodnotu jsme určili experimentálně jako 0,2. Z grafů \ref{ram_p_1} a \ref{ram_p_-1} je vidět, že překmit ramene je přibližně 10\% a rameno se ustálí v požadované hodnotě přibližně za 1 sekundu. Akční zásah, viz. graf. \ref{ram_p_akc}, sice z počátku překročí saturační hodnotu tj. 1, což však, jak jde vidět z hodnot grafu \ref{ram_p_odz}, nevadí, negativní vliv na překmit to nemá.

% P 1 
\begin{figure}[H]
\centering
\includegraphics[width=0.7\textwidth]{dobre_grafy/rameno_odezva_na1_P_akcnizasah.eps}
\caption{rameno odezva na1 P akcnizasah.eps}
\label{ram_p_1}
\end{figure}

\begin{figure}[H]
\centering
\includegraphics[width=0.7\textwidth]{dobre_grafy/rameno_odezva_na1_P.eps}
\caption{rameno odezva na1 P.eps}
\label{ram_p_1_akc}
\end{figure}
% P 1 -- end

% P -1
\begin{figure}[H]
\centering
\includegraphics[width=0.7\textwidth]{dobre_grafy/rameno_odezva_a_porucha_P_-1.eps}
\caption{rameno odezva a porucha P -1.eps}
\label{ram_p_-1}
\end{figure}

\begin{figure}[H]
\centering
\includegraphics[width=0.7\textwidth]{dobre_grafy/rameno_odezva_a_porucha_P_akcnizasah_-1.eps}
\caption{rameno odezva a porucha P akcnizasah -1.eps}
\label{ram_p_-1_akc}
\end{figure}
% P -1 -- end


\subsection{První obecný regulátor 2. řádu}
Pomocí nástroje \textit{rltool} v MATLABu jsme nastavili póly uzavřené smyčky tak, aby se systém ustaloval co nejrychleji s dostatečně malým překmitem (viz. obrázek níže), ale na nelineárním modelu se ukázalo, že překmit je podstatně větší, než při simulaci na linearizovaném modelu, což se pak potvrdilo také na reálném modelu. Jinak ale tento regulátor funguje celkem rychle a uspokojivě. Výsledný přenos je:
$$
C = \frac{1,61 (s+0,19)}{ s (s+9,43)}
$$
\newline
Pro příklady odezev, viz. obrázky níže.
% leadlagmozna -1
\begin{figure}[H]
\centering
\includegraphics[width=0.7\textwidth]{rltool_rameno1.eps}
\caption{Návrh regulátoru polohy ramene v rltoolu}
\label{ram_rltool1}
\end{figure}
\begin{figure}[H]
\centering
\includegraphics[width=0.7\textwidth]{dobre_grafy/rameno_odezva_a_porucha_leadlagmozna_-1.eps}
\caption{rameno odezva a porucha leadlagmozna -1.eps}
\label{ram_ob1}
\end{figure}

\begin{figure}[H]
\centering
\includegraphics[width=0.7\textwidth]{dobre_grafy/rameno_odezva_a_porucha_leadlagmozna_akcnizasah_-1.eps}
\caption{rameno odezva a porucha leadlagmozna akcnizasah -1.eps}
\label{ram_ob1_akc}
\end{figure}
% leadlagmozna -1 -- end

\subsection{Druhý obecný regulátor 2. řádu}
Protože první regulátor, navržený rltoolem měl příliš velký překmit, zkusili jsme návrh zlepšít, což se nám zčásti podařilo. Tento další regulátor má sice podstatně menší překmit a srovnatelnou dobu ustálení, ale neustaluje se tak docela na nulové odchylce. Uvedli jsme zde proto oba regulátory, protože každý může být vhodnější pro jinou situaci. Přenos regulátoru je:
$$
C = \frac{3,05 (s+0,11)}{s (s+22,1)}
$$
\newline
Pro ukázku návrhu v rltoolu a příklady odezev, viz. obrázky níže.
% leadlagmozna2 -1
\begin{figure}[H]
\centering
\includegraphics[width=0.7\textwidth]{rltool_rameno2.eps}
\caption{Návrh regulátoru polohy ramene v rltoolu}
\label{ram_rltool2}
\end{figure}
\begin{figure}[H]
\centering
\includegraphics[width=0.7\textwidth]{dobre_grafy/rameno_odezva_a_porucha_leadlagmozna2_-1.eps}
\caption{rameno odezva a porucha leadlagmozna2 -1.eps}
\label{ram_ob2}
\end{figure}

\begin{figure}[H]
\centering
\includegraphics[width=0.7\textwidth]{dobre_grafy/rameno_odezva_a_porucha_leadlagmozna2_akcnizasah-1.eps}
\caption{rameno odezva a porucha leadlagmozna2 akcnizasah-1.eps}
\label{ram_ob2_akc}
\end{figure}
% leadlagmozna2 -1 -- end


\section{Regulátor úhlu natočení kyvadla}
Regulátor pro kyvadlo jsme navrhovali tak, aby co nejrychleji a s co nejmenším překmitem reguloval polohu kyvadla vždy do spodní polohy, tedy na hodnotu $\varphi_p=-90^{\circ}=-\frac{\pi}{2}$. Každý regulátor jsme vždy nejdříve navrhli nějakou metodou na linearizovaném systému v pracovním bodě $\mathbf{x_0} = \left[\varphi_{m0}, \varphi_{p0}, \dot{\varphi}_{m0}, \dot{\varphi}_{p0}\right] = \left[0, -\frac{\pi}{2}, 0, 0\right]$, potom jsme ho vyzkoušeli na linearizovaném a nelineárním modelu a zkontrolovali jsme, že se chová rozumně - především jsme kontrolovali, že vstupní saturace systému nepokazí odezvy. Až pokud regulátor obstál na linearizovaném i nelineárním modelu, vyzkoušeli jsme ho na reálném systému. Zapojení pro testování regulátorů je na následujícím obrázku:
\begin{figure}[H]
	\centering
    \includegraphics[scale=0.25]{schema_rizeni_nanecisto}
    \caption{Testovací zapojení regulátorů}
\end{figure}

Při regulování kyvadla jsme se setkali s problémem, který se nám nepodařilo rozumně odstranit, ani odejít. Senzor polohy kyvadla totiž někdy vynechával (což mohlo být způsobeno senzorem kyvadla, nebo nestabilní komunikací mezi MATLABem a řídicí elektronikou), a takto vzniklá chyba se integrovala, kvůli čemuž se někdy neustálila odchylka systému na nule, i když kyvadlo bylo ve svislé poloze, a rameno se stále točilo. Tuto chybu senzoru lze dobře vidět na následujícím grafu odezvy kyvadla na poruchu bez regulátoru (dva "skoky" jinak pseudospojitého signálu). Samotné kyvadlo se jinak ustálí za velmi dlouhou dobu - až 20 sekund.
\begin{figure}[H]
	\centering
    \includegraphics[scale=0.55]{odezva_kyvadlo}
    \caption{Odezva kyvadla na poruchu s viditelnou chybou senzoru}
\end{figure}

\subsection{Obecný regulátor, navržený polynomiálními metodami}
Tento regulátor jsme navrhovali polynomiálními metodami. Nejdříve jsme si rozumně určili polohy, do kterých bychom chtěli posunout póly výsledného systému a sestavili charakteristický polynom $c(s)$ pro tyto póly. Potom jsme dosadili do rovnice pro charakteristický polynom výsledného systému se zapojeným regulátorem, kde regulátor $C = \frac{y(s)}{x(s)}$ a soustava $G = \frac{b(s)}{a(s)}$, a tuto rovnici se dvěma neznámými polynomy $y$ a $x$ vyřešili pomocí funkce Polynomial Toolboxu \textit{axbyc} s parametrem \textit{"miny"}:
\begin{align*}
c(s) &= (s + 12) (s + 5 - 10j) (s + 5 + 10j) (s + 20)^2	\\
b(s)\cdot y(s) + a(s)\cdot x(s) &= c(s)	\\
-18,1s\cdot y(s) + (s^3 + 7,021s^2 + 74,37s + 461)\cdot x(s) &=  (s + 12) (s + 5 - 10j) (s + 5 + 10j) (s + 20)^2
\end{align*}
Do žádaného charakteristického polynomu jsme museli přidat dva póly ($(s+20)^2$), aby měla tato rovnice řešení, které vede na ryzí regulátor.
\newline
Výsledný přenos regulátoru:
\begin{align*}
C = \frac{13,09s^2 - 354,3s - 1981}{16s^2 + 879,7 + 20820}
\end{align*}
Příklad odezvy kyvadla s tímto regulátorem na poruchu je na grafech níže. Z grafů je vidět, že kyvadlo se neustaluje na nulové odchylce, což je způsobeno zmiňovanou chybou senzoru.
\newline
\newline
Až na tuto chybu ale regulátor funguje velmi dobře a kyvadlo stabilizuje vždy maximálně do dvou sekund.
\begin{figure}[H]
	\centering
    \includegraphics[scale=0.55]{odezva_kyvadlo_poly}
    \caption{Odezva kyvadla s polynomiálním regulátorem na poruchu}
\end{figure}
\begin{figure}[H]
	\centering
    \includegraphics[scale=0.55]{odezva_kyvadlo_poly_akcnizasah}
    \caption{Akční zásah polynomiálního regulátoru}
\end{figure}


\subsection{PID regulátor, navržený autotunem}
Tento regulátor jsme navrhli po několika neúspěšných pokusech v rltoolu navrhnout nějaký rozumný PID regulátor pomocí funkce autotune. Parametry jsme nastavili na PID tuning, robustní časovou odezvu, PID s filtrovanou D složkou prvního řádu a vyváženou robustnost a výkon.
\newline
Obecná rovnice PID regulátoru a hodnoty, navržené autotunem:
\begin{align*}
C &= K_P + K_I\frac{1}{s} + K_D\frac{N}{1+N\frac{1}{s}}	\\
C &= 2,09 + 7,91\frac{1}{s} + 0,121\frac{238}{1 + 238\frac{1}{s}}
\end{align*}
Přenos, navržený autotunem:
\begin{align*}
C = -100,67\cdot\frac{(s+10,68)(s+5,715)}{s(s+238)}
\end{align*}
Následují jsou grafy odezvy kyvadla na poruchu. Tento regulátor stabilizuje velmi rychle - okolo jedné sekundy, ale kvůli nedokonalosti modelu, který nepočítá s vlivem rychlosti ramene na výchylku kyvadla, ustálenému akčnímu zásahu, který není nulový, a také kvůli chybě senzoru, se systém s regulátorem po delší době (asi 20 sekund) destabilizuje.
\begin{figure}[H]
	\centering
    \includegraphics[scale=0.55]{odezva_kyvadlo_PID}
    \caption{Odezva kyvadla s PID regulátorem na poruchu}
\end{figure}
\begin{figure}[H]
	\centering
    \includegraphics[scale=0.55]{odezva_kyvadlo_PID_akcnizasah}
    \caption{Akční zásah PID regulátoru}
\end{figure}

\section{Závěr}
Z námi navržených regulátorů nejlépe fungoval pro rameno regulátor, navržený pomocí rltoolu, a pro kyvadlo regulátor, navržený polynomiálními metodami. Ale i jednoduchý P regulátor polohy ramene nás překvapil tím, jak dobře fungoval. Na druhou stranu PID regulátor kyvadla se příliš neosvědčil hlavně kvůli své tendenci se postupně destabilizovat, i když jinak reguluje velmi rychle a s malým překmitem. Také jsme navrhli několik dalších regulátorů, které měly své výhody, jako například P regulátor pro kyvadlo (se zesílením $P=-0.2$), který fungoval také překvapivě dobře a neměl problém s "ujížděním", jako PID, nebo další obecné regulátory pro rameno, které často regulovaly velmi rychle a bez překmitu, ale měly problém s ustálením se na nenulové odchylce kvůli zóně necitlivosti vstupu, proto jsme je nevybrali mezi uvedené.
\newline

\end{document}