\documentclass[11pt,a4paper]{article}
\usepackage[czech]{babel}
\usepackage[utf8]{inputenc}
\usepackage{amsmath}
\usepackage{float} %kvůli umístění tabulek a podobně.
%\usepackage[]{mcode} %pro matlabovský kód
\usepackage{etoolbox} %kvůli strequal v newcommand



\usepackage[czech]{babel}
\usepackage[utf8]{inputenc}

\usepackage{graphicx}	%kvůli pdf souboru z MATLABu
\usepackage{caption}		%kvůli umístění několika obrázku do jednoho figuru
\usepackage{subcaption} %kvůli umístění několika obrázku do jednoho figuru
\usepackage{setspace}	%rozteče v maticích
\usepackage{verbatim}   %víceřádkové komentáře
\usepackage{siunitx}	%hezký jednotky

\sisetup{locale = DE} % kvůli desetinné čárce (př. v siunits.)


% === Formát stránky ===
\usepackage[a4paper]{geometry}
\geometry{
	verbose,
	tmargin=2.2cm,
	bmargin=1.5cm,
	lmargin=1.5cm,
	rmargin=1.5cm}

\title{%
  Laboratorní úloha - řídící část\\
  \large Rotační kyvadlo}
\author{Tomáš Glabazňa, Matouš Vrba}
\date{\today}

\begin{document}
\maketitle

\clearpage

\section{Proporční regulátor}

Proporční zpětnovazební regulátor jsme navrhovali přímo na modelu a jeho vhodnou hodnotu jsme určili experimentálně jako 0,2. Z grafu \ref{ram_p_odz} je vidět, že překmit ramene je přibližně 10\% a rameno se ustálí v požadované hodnotě přibližně za 1 sekundu. Akční zásah, viz. graf. \ref{ram_p_akc}, sice z počátku překročí saturační hodnotu tj. 1,  

\begin{figure}[H]
\centering
\includegraphics[width=0.7\textwidth]{rameno_odezva_10_P}
\caption{Odezva ramene na referenci 10}
\label{ram_p_odz}
\end{figure}

\begin{figure}[H]
\centering
\includegraphics[width=0.7\textwidth]{rameno_odezva_10_P_akcnizasah}
\caption{Akční zásah proporčního regulátoru při referenci 10}
\label{ram_p_akc}
\end{figure}
\end{document}